% \iffalse meta-comment
%
% File: xistercian.dtx Copyright (C) 2021 Jonathan P. Spratte
%
% This work  may be  distributed and/or  modified under  the conditions  of the
% LaTeX Project Public License (LPPL),  either version 1.3c  of this license or
% (at your option) any later version.  The latest version of this license is in
% the file:
%
%   http://www.latex-project.org/lppl.txt
%
% ------------------------------------------------------------------------------
%
%<*driver>^^A>>=
\def\nameofplainTeX{plain}
\ifx\fmtname\nameofplainTeX\else
  \expandafter\begingroup
\fi
\input l3docstrip.tex
\askforoverwritefalse
\preamble

--------------------------------------------------------------
xistercian -- Cistercian numerals in LaTeX
E-mail: jspratte@yahoo.de
Released under the LaTeX Project Public License v1.3c or later
See http://www.latex-project.org/lppl.txt
--------------------------------------------------------------

Copyright (C) 2020-2021 Jonathan P. Spratte

This  work may be  distributed and/or  modified under  the conditions  of the
LaTeX Project Public License (LPPL),  either version 1.3c  of this license or
(at your option) any later version.  The latest version of this license is in
the file:

  http://www.latex-project.org/lppl.txt

This work is "maintained" (as per LPPL maintenance status) by
  Jonathan P. Spratte.

This work consists of the file  xistercian.dtx
and the derived files           xistercian.pdf
                                xistercian.sty

\endpreamble
% stop docstrip adding \endinput
\postamble
\endpostamble
\generate{\file{xistercian.sty}{\from{xistercian.dtx}{pkg}}}
\ifx\fmtname\nameofplainTeX
  \expandafter\endbatchfile
\else
  \expandafter\endgroup
\fi
%
\IfFileExists{xistercian.sty}{\RequirePackage[debug]{xistercian}}{}
\ProvidesFile{xistercian.dtx}[2021-12-20 v1.2 Cistercian numerals in LaTeX]
\PassOptionsToPackage{full}{textcomp}
\documentclass{l3doc}
\RequirePackage[fulloldstyle,nott]{kpfonts}
\RequirePackage{microtype}
\RequirePackage{csquotes}
\RequirePackage{tikz}
\usetikzlibrary{arrows.meta}
\RequirePackage{randtext}
\input{glyphtounicode}
\pdfgentounicode=1
\makeatletter
\@ifdefinable\gobbledocstriptag{\def\gobbledocstriptag#1>{}}
\makeatother
\let\metaORIG\meta
\protected\def\meta#1{\texttt{\metaORIG{#1}}}
\renewcommand*\thefootnote{\cistercian{footnote}}
\newcommand*\cistexnum[1]
  {^^A
    \begingroup
      \color{red!65!black}^^A
      \cisterciansetup{s=.15ex}^^A
      \cisterciannum{#1}^^A
    \endgroup
  }
\pagenumbering{cistercian}
\ExplSyntaxOn
\makeatletter
\AddToHook{env/syntax/begin}{\let\myhfill\hfill}
\let\myhfill\@empty
\newcommand*\mydefault[1]
  {
    \begingroup
      \color{darkgray}
      \myrestoredefaults
      \myhfill
      \texttt
        {
          \expandafter\ifx\csname xister@#1\endcsname\@empty
            \normalfont\emph{empty}
          \else
            \detokenize\expandafter\expandafter\expandafter
              {\csname xister@#1\endcsname}
            \unskip
          \fi
        }
    \endgroup
  }
\newcommand*\mysdefault[1]
  {
    \begingroup
      \color{darkgray}
      \myrestoredefaults
      \myhfill
      \texttt { \str_if_eq:nnTF {#1} { empty } { \normalfont \emph{#1} } {#1} }
    \endgroup
  }
\def\myrestoredefaults#1{\def\unexpanded{#1}{\unexpanded\expandafter{#1}}}
\edef\myrestoredefaults
  {
    \myrestoredefaults\xister@x
    \myrestoredefaults\xister@y
    \myrestoredefaults\xister@bx
    \myrestoredefaults\xister@by
    \myrestoredefaults\xister@tv
    \myrestoredefaults\xister@th
    \myrestoredefaults\xister@tdu
    \myrestoredefaults\xister@tdd
    \myrestoredefaults\xister@ts
    \myrestoredefaults\xister@font
  }
\makeatother
\newcommand\examplerow[1]
  {
    \cisterciannum{1#1}&\cisterciannum{2#1}&\cisterciannum{3#1}&
    \cisterciannum{4#1}&\cisterciannum{5#1}&\cisterciannum{6#1}&
    \cisterciannum{7#1}&\cisterciannum{8#1}&\cisterciannum{9#1}\\
    \noalign{\vskip-.5ex}
    $1#1$&$2#1$&$3#1$&
    $4#1$&$5#1$&$6#1$&
    $7#1$&$8#1$&$9#1$\\
    \noalign{\vskip1ex}
  }
\AddToHook{bfseries}
  {\cisterciansetup{sv=.27ex,sdd=.27ex,sh=.12ex,sdu=.12ex,wd=.625em}}
\AddToHook{mdseries}{\myrestoredefaults}
\cisterciansetup{redraw+=series}
\tl_new:N \l__xisterexample_digit_tl
\newcommand*\showbb[1]
  {
    \begingroup
      \cisterciansetup{#1}\cistercianredraw
      \fboxrule=.5pt
      \fboxsep=-.5\fboxrule
      \leavevmode
      \rlap
        {
          \textcolor{red}
            {\fbox{\phantom{\cisterciannumE{\l__xisterexample_digit_tl}}}}
        }
      \kern.5\fboxrule
      \cisterciannumE{\l__xisterexample_digit_tl}
      \kern.5\fboxrule
    \endgroup
  }
\newcommand*\exampledigits{\tl_set:Nn \l__xisterexample_digit_tl}
\cistercianstyle{.ex}{sv=3pt,sdd=3pt,sh=2pt,sdu=2pt}
\cistercianstyle{vex}{o=v,bx=,by=,wd=7mm,ht=14mm,.ex}
\cistercianstyle{hex}{o=h,bx=,by=,ht=7mm,wd=14mm,.ex}
\ExplSyntaxOff
\begin{document}
  \title{The \pkg{xistercian} package}
  \author{Jonathan P. Spratte\thanks{\protect\randomize{jspratte@yahoo.de}}}
  \date{2021-12-20 v1.2}
  \DocInput{xistercian.dtx}
\end{document}
%</driver>^^A=<<
% \fi
%
% \maketitle
% \tableofcontents
%
% \section{Documentation}
%
% The \pkg{xistercian} package provides Cistercian numerals for use in \LaTeX.
% The name is chosen to be \pkg{xistercian} instead of \pkg{cistercian} because
% I've learned that a (currently) unpublished package by that name already
% exists and I don't want to be an evil name-stealer.
%
%
% \subsection{Introduction}
%
% Cistercian numerals are a system to denote the numbers from $1$ to $9999$ with
% a single glyph. They use a stem (\cistexnum{0}) that is used as zero by this
% package, and add the digits to that using small tick marks, the place denoting
% the digits' value. On the upper right means units (\cistexnum{1}), upper left
% tens (\cistexnum{10}), lower right hundreds (\cistexnum{100}), and lower left
% thousands (\cistexnum{1000}). Figure~\ref{fig:glyphs} gives an overview over
% the base glyphs.
%
% In the medieval times it was quite common (according to Wikipedia more common
% than the vertical style) to use the Cistercian numerals horizontally instead.
% This package also supports that, in which case the numerals are rotated by
% $90$ degrees counter-clockwise, see figure~\ref{fig:glyphs:h}.
%
% Moreover some glyphs have a different shape in some manuscripts. The shapes in
% figures~\ref{fig:glyphs} and~\ref{fig:glyphs:h} are the most used ones
% nowadays (most likely because those are the ones Wikipedia shows vector
% graphics of). This package offers some (not all) alternative forms. Most
% notably some historic documents have swapped shapes of digits. To not further
% the confusion this package doesn't support swapping digit shapes.
%
% All glyphs have the same bounding box, are drawn using the \pkg{pgf}
% package, and cached inside \TeX\ box registers. Therefore the performance is
% quite well, at least well enough to actually consider using these numerals
% without a major performance hit. But it still slows down the compilation,
% especially if the glyphs have to be redrawn often. By default they will be
% redrawn if the font size changed since the last usage inside the current group
% scope, but you can specify which font feature changes should cause a redraw.
%
% \begin{figure}
%   \centering
%   \small
%   \cistercianredraw
%   \begin{tabular}{*9c}
%     \examplerow{}
%     \examplerow{0}
%     \examplerow{00}
%     \examplerow{000}
%   \end{tabular}
%   \caption{The different base glyphs in vertical mode\label{fig:glyphs}}
% \end{figure}
%
% \begin{figure}
%   \centering
%   \small
%   \cisterciansetup{horizontal}\cistercianredraw
%   \begin{tabular}{*9c}
%     \examplerow{}
%     \examplerow{0}
%     \examplerow{00}
%     \examplerow{000}
%   \end{tabular}
%   \caption{The different base glyphs in horizontal mode\label{fig:glyphs:h}}
% \end{figure}
%
% Since Cistercian numerals quite compactly cover a big range of natural numbers
% they might be used to compactly keep track of running numbers. For instance,
% to use Cistercian numerals as page numbers you can use:
% \begin{verbatim}
% \pagenumbering{cistercian}
% \end{verbatim}
% or to use them as footnotes:
% \begin{verbatim}
% \renewcommand\thefootnote{\cistercian{footnotes}}
% \end{verbatim}
%
% The package tries to play nice on \pkg{hyperref} if it is loaded, but
% bookmarks containing material that should've been a Cistercian numeral will
% have a leading |CISTER| in front of the numeric value to give unique
% strings compared to |\arabic| (to my knowledge there are no Unicode points for
% Cistercian numerals).
%
% Negative numbers just keep a leading~|-|, and numbers with an absolute number
% greater than $9999$ are displayed with multiple Cistercian digits. For
% instance $-12345678$ is displayed as \cistexnum{-12345678}. And in the case
% that four consecutive digits are zeros they get displayed as that:
% \cistexnum{10000} is $10000$.
%
% Full disclaimer: I'm neither a historian nor have I any authority on the
% matter. I'm just a bloke who was fascinated by this numeral system and wanted
% to provide \LaTeX\ support for it. Most of my knowledge about this stems from
% Wikipedia and similar sources.
%
%
% \subsection{Macros}
%
% \begin{function}{\cistercian}
%   \begin{syntax}
%     \cs{cistercian}\marg{counter}
%   \end{syntax}
%   Prints the value of the \LaTeXe\ \meta{counter} as a Cistercian numeral
%   (similar to how |\roman| prints a \meta{counter} as a Roman numeral). It can
%   also be used as |\pagenumbering|\hskip0pt |{cistercian}| to change the page
%   numbering, \emph{etc}.
% \end{function}
%
% \begin{function}{\cisterciannum}
%   \begin{syntax}
%     \cs{cisterciannum}\marg{integer}
%   \end{syntax}
%   Prints the \meta{integer} as a Cistercian numeral. The \meta{integer} has to
%   be provided as a string containing only digits and optionally a single
%   leading~|-| (spaces are ignored). This can handle integers of arbitrary
%   size.
% \end{function}
%
% \begin{function}{\cisterciannumE}
%   \begin{syntax}
%     \cs{cisterciannumE}\marg{integer expression}
%   \end{syntax}
%   Evaluates the \meta{integer expression} and prints the result as a
%   Cistercian numeral.
% \end{function}
%
% \begin{function}{\cisterciansetup}
%   \begin{syntax}
%     \cs{cisterciansetup}\marg{key=value,\ldots}
%   \end{syntax}
%   Can be used to locally change the options after the package was loaded. The
%   glyphs of the Cistercian numerals aren't automatically updated when you
%   change the setup.
% \end{function}
%
% \begin{function}{\cistercianredraw,\cistercianredrawlazy}
%   \begin{syntax}
%     \cs{cistercianredraw}
%   \end{syntax}
%   When this is used the glyphs will be redrawn for the current group. The
%   |lazy| variant will not immediately redraw the glyphs, but instead ensure
%   that the next usage of a Cistercian numeral in the current scope will redraw
%   the glyphs. The |lazy| variant could result in the glyphs being redrawn
%   multiple times (if Cistercian numerals are used in different nested groups),
%   while the normal variant might result in the glyphs being redrawn too
%   eagerly.
% \end{function}
%
% \begin{function}{\cistercianstyle}
%   \begin{syntax}
%     \cs{cistercianstyle}\meta{*}\marg{name}\marg{key=value,\ldots}
%   \end{syntax}
%   With this macro you can define a key called \meta{name} that'll set the
%   options in the \meta{key=value} list if used inside |\cisterciansetup|.
%   Only new names are allowed. If your style shouldn't take an argument (so is
%   fixed) use the normal variant. If your style needs to take an argument use
%   the variant with \meta{*}.
% \end{function}
% For example, the key |horizontal| is equivalently defined to and can be used
% as
% \begin{verbatim}
% \cistercianstyle{horizontal}{o=h,wd=0.775em,ht=1.13ex}
% \cisterciansetup{horizontal}
% \end{verbatim}
% And if you want to define a style that at the same time changes the stroke
% widths for vertical and horizontal strokes you could use
% \begin{verbatim}
% \cistercianstyle*{sh+v}{sh={#1},sv={#1}}
% \cisterciansetup{sh+v=.12ex}
% \end{verbatim}
%
%
% \subsection{Options}
%
% The options described here are all usable as package options (except when
% explicitly stated otherwise). \pkg{xistercian} doesn't parse the global
% options provided to the documentclass. Additionally you can change the
% options (except for the |debug| option) using |\cisterciansetup|.
%
% For the options |width| and |height|, and probably also the strokes, you
% should use dimensions using |ex| or |em| to get sizes depending on the current
% font size. A \meta{dimension expression} is evaluated (using the setup
% specified in |font|) every time the font changed according to the |redraw|
% option's feature list (by default only when the size changed) since the last
% time they were drawn.
%
% Initial values if present are printed on the right. A small graphic explaining
% most of the size related options is shown in figure~\ref{fig:size}.
%
% \begin{figure}
%   \centering
%   \def\share{1/3}
%   \expanded
%     {^^A
%       \cisterciansetup
%         {bx=1,by=1,fr=\share,ht=4cm,wd=2cm,sv=9pt,sh=5pt,sdd=9pt,sdu=5pt}^^A
%     }^^A
%   \cistercianredraw
%   \begin{tikzpicture}[>=Stealth,baseline=(base)]
%     \node[color=green,inner sep=0] (sym) {\cisterciannum{9}};
%     \begin{scope}[y=(sym.north),x=(sym.east)]
%       \draw (-1,-1) rectangle (1,1);
%       \draw[dashed, line width=0.2pt]
%         \foreach\x in{{-1,0},{1,0},{0,-1},{0,1}} { (0,0) -- (\x) }
%         ;
%       \draw[dashed, line width=0.2pt] (0, \share) coordinate(fr) -- ++(1,0);
%       \draw[<->]
%         (1.2, 0) -- node[rotate=90,below]{\texttt{0.5*fr*ht}} (1.2,0 |- fr);
%       \draw[<->] (-1, -1.1) -- node[below]{\texttt{wd}} ++(2,0);
%       \draw[<->] (-1.2, -1) -- node[rotate=90,above]{\texttt{ht}} ++(0,2);
%       \draw[red] (-1.6, -1.3) rectangle (1.6, 1.3);
%       \draw[<->] (-1.6, -1.4) -- node[below]{\texttt{bx*wd}} (1.6, -1.4);
%       \draw[<->]
%         (1.8, -1.3) -- node[rotate=90,below]{\texttt{by*ht}} (1.8, 1.3);
%       \draw[<->]
%         (-1.8, 0) -- node[rotate=90,above]{\texttt{0.5*bs*ht}} (-1.8, -.75);
%       \fill (-1.6, -0.75) circle[radius=2pt] coordinate(base);
%     \end{scope}
%   \end{tikzpicture}
%   \cisterciansetup{o=h,ht=2cm,wd=4cm}\cistercianredraw
%   \begin{tikzpicture}[>=Stealth,baseline=(base)]
%     \node[color=green,inner sep=0] (sym) {\cisterciannum{9}};
%     \begin{scope}[y=(sym.north),x=(sym.east)]
%       \draw (-1,-1) rectangle (1,1);
%       \draw[dashed, line width=0.2pt]
%         \foreach\x in{{-1,0},{1,0},{0,-1},{0,1}} { (0,0) -- (\x) }
%         ;
%       \draw[dashed, line width=0.2pt] (-\share,0) coordinate(fr) -- ++(0,1);
%       \draw[<->] (0, 1.2) -- node[above]{\texttt{0.5*fr*wd}} (0,1.2 -| fr);
%       \draw[<->] (-1, -1.2) -- node[below]{\texttt{wd}} ++(2,0);
%       \draw[<->] (-1.1, -1) -- node[rotate=90,above]{\texttt{ht}} ++(0,2);
%       \draw[red] (-1.3, -1.6) rectangle (1.3, 1.6);
%       \draw[<->] (-1.3, -1.8) -- node[below]{\texttt{bx*wd}} (1.3, -1.8);
%       \draw[<->]
%         (1.4, -1.6) -- node[rotate=90,below]{\texttt{by*ht}} (1.4, 1.6);
%       \draw[<->]
%         (-1.4, 0) -- node[rotate=90,above]{\texttt{0.5*bs*ht}} (-1.4, -.75);
%       \fill (-1.3, -0.75) circle[radius=2pt] coordinate(base);
%     \end{scope}
%   \end{tikzpicture}
%   \caption[Measurements of a glyph]
%     {Measurements of a glyph. The red rectangle is the bounding box, the
%     dot on the left shows the reference point placed on the surrounding
%     baseline.\label{fig:size}}
% \end{figure}
%
% \begin{function}{width,wd}
%   \begin{syntax}
%     width = \marg{dimension expression} \mydefault{x}
%   \end{syntax}
%   Set the width of the Cistercian numeral digits.
% \end{function}
%
% \begin{function}{height,ht}
%   \begin{syntax}
%     height = \marg{dimension expression} \mydefault{y}
%   \end{syntax}
%   Set the height of the Cistercian numeral digits.
% \end{function}
%
% \begin{function}{bound-x,bx}
%   \begin{syntax}
%     bound-x = \marg{float} \mydefault{bx}
%   \end{syntax}
%   Specifies the factor the bounding box in $x$-direction is bigger than the
%   |width|. An empty \meta{float} is considered the same as $1$.
% \end{function}
%
% \begin{function}{bound-y,by}
%   \begin{syntax}
%     bound-y = \marg{float} \mydefault{by}
%   \end{syntax}
%   Specifies the factor the bounding box in $y$-direction is bigger than the
%   |height|. An empty \meta{float} is considered the same as $1$.
% \end{function}
%
% \begin{function}{baseline,bs}
%   \begin{syntax}
%     baseline = \marg{float} \mydefault{bs}
%   \end{syntax}
%   Sets the baseline of the symbols in multiples of the negative height. $1$
%   (or empty) results in the symbol starting on the surrounding baseline,
%   bigger values shift the symbols upwards, smaller values downwards.
% \end{function}
%
% \begin{function}{stroke-v,sv}
%   \begin{syntax}
%     stroke-v = \marg{dimension expression} \mydefault{tv}
%   \end{syntax}
%   Sets the stroke thickness of vertical strokes.
% \end{function}
%
% \begin{function}{stroke-h,sh}
%   \begin{syntax}
%     stroke-h = \marg{dimension expression} \mydefault{th}
%   \end{syntax}
%   Sets the stroke thickness of horizontal strokes.
% \end{function}
%
% \begin{function}{stroke-du,sdu}
%   \begin{syntax}
%     stroke-du = \marg{dimension expression} \mydefault{tdu}
%   \end{syntax}
%   Sets the stroke thickness of diagonal strokes going from lower left to upper
%   right.
% \end{function}
%
% \begin{function}{stroke-dd,sdd}
%   \begin{syntax}
%     stroke-dd = \marg{dimension expression} \mydefault{tdd}
%   \end{syntax}
%   Sets the stroke thickness of diagonal strokes going from upper left to lower
%   right.
% \end{function}
%
% \begin{function}{stroke-s,ss}
%   \begin{syntax}
%     stroke-s = \marg{dimension expression} \mydefault{ts}
%   \end{syntax}
%   Sets the stroke thickness of the zero stem. If this is set to an empty value
%   the stroke thickness in vertical mode is that of |stroke-v| and in
%   horizontal mode that of |stroke-h|.
% \end{function}
%
% \begin{function}{strokes,s}
%   \begin{syntax}
%     stroke = \marg{dimension expression}
%   \end{syntax}
%   Sets all stroke thicknesses at once.
% \end{function}
%
% \begin{function}{fraction,fr}
%   \begin{syntax}
%     fraction = \marg{float} \mysdefault{/6}
%     fraction = \marg{float}/\marg{integer}
%   \end{syntax}
%   The Cistercian digits are drawn only in a fraction of the total bounding box
%   of each symbol. The ones are drawn (with vertical Cistercians) in the upper
%   right rectangle. The width of that rectangle is determined by |width|, and
%   the upper border by |height|, and with this option you can set the lower
%   border as a fraction of the |height|. You can either just give a factor as a
%   \meta{float}, or just pass in a divisor as an \meta{integer} (with a leading
%   slash), or both. An empty \meta{float} equals $1$.
% \end{function}
% Consider the following example (result on the right):\\[\bigskipamount]
% \begin{minipage}[c]{.8\linewidth}
% \begin{verbatim}
% {\cisterciansetup{fr=2/9,redraw}\cisterciannum{9}}\par
% {\cisterciansetup{fr=0.6,redraw}\cisterciannum{9}}\par
% \end{verbatim}
% \end{minipage}
% \begin{minipage}[c]{1em}
% {\cisterciansetup{fr=2/9,redraw}\cisterciannum{9}}\par
% {\cisterciansetup{fr=0.6,redraw}\cisterciannum{9}}\par
% \end{minipage}
%
% \begin{function}{orientation,o}
%   \begin{syntax}
%     orientation = \marg{choice} \mysdefault{vertical}
%   \end{syntax}
%   Sets the orientation of the Cistercian numerals. Choices are |horizontal|
%   (or |h| for short) or |vertical| (or |v| for short). You'll most likely want
%   to change the |width| and |height| of the symbols if you decide to change
%   this.
% \end{function}
%
% \begin{function}{horizontal}
%   \begin{syntax}
%     horizontal
%   \end{syntax}
%   Same as setting
%   |orientation = horizontal, width = 0.775em, height = 1.13ex|.
% \end{function}
%
% \begin{function}{vertical}
%   \begin{syntax}
%     vertical
%   \end{syntax}
%   Same as setting |orientation = vertical, width = 0.5em, height = 1.55ex|.
% \end{function}
%
% \begin{function}{alternate-5,5}
%   \begin{syntax}
%     alternate-5 = \marg{choice} \mysdefault{triangle}
%   \end{syntax}
%   There are alternative forms of the digit $5$. While the most often used one
%   nowadays seems to be the |triangle| form, this package also supports two
%   other variants. Those are called |dot| and |dash|. While if you choose
%   |triangle| you can't give an additional value, if your choice is |dot| or
%   |dash| you can customise those using a second equals sign and some value. If
%   you don't customise them they use their respective initial value (or last
%   value if you changed the value at some point in time).
%
%   For |dot| you can also specify the radius as a dimension. The default is
%   \mydefault{five@dot}.
%
%   For |dash| you can specify the length of the dash in multiples of the used
%   stroke thickness (which might differ for the horizontal and vertical
%   symbols), an empty value is considered the same as $1$. The default is
%   \mydefault{five@dash}. Please ensure that your readers can't confuse the
%   results with the shape of $6$.
%
%   A comparison of the three alternate forms is shown in
%   figure~\ref{fig:fives}.
% \end{function}
% The usage might look like any of the following:
% \begin{verbatim}
% \cisterciansetup
%   {
%     5=triangle, 5={triangle},
%     5=dot,  5={dot},  5=dot=.1ex, 5={dot=.1ex},
%     5=dash, 5={dash}, 5=dash=2,   5={dash=2}
%   }
% \end{verbatim}
%
% \begin{figure}
%   \centering
%   \exampledigits{5555}%
%   \begingroup
%     \cisterciansetup{vex}%
%     \begin{tabular}{*5c}
%       \texttt{triangle}        &
%         \texttt{dot=2pt}       & \texttt{dot=4pt} &
%         \texttt{dash=1}        & \texttt{dash=3}  \\
%       \showbb{5=triangle}      &
%         \showbb{5=dot=2pt}     & \showbb{5=dot=4pt} &
%         \showbb{5=dash=1}      & \showbb{5=dash=3}  \\
%       \showbb{5=triangle,hex}  &
%         \showbb{5=dot=2pt,hex} & \showbb{5=dot=4pt,hex} &
%         \showbb{5=dash=1, hex} & \showbb{5=dash=3, hex} \\
%     \end{tabular}
%   \endgroup
%   \caption[Alternate forms of the digit $5$]
%     {Alternate forms of the digit $5$. The length of the \texttt{dash=3}
%      variant is too long in the normal setup of this document, as the
%      difference between {\cisterciansetup{5=dash=3,r!}\cisterciannum{5555}}
%      and \cisterciannum{6666} is hard to notice. The tight bounding boxes
%      using \texttt{bx=1,by=1} are drawn red.\label{fig:fives}}
% \end{figure}
%
% \begin{function}{alternate-6,6}
%   \begin{syntax}
%     \cs{alternate-6} = \marg{choice} \mysdefault{stroke}
%   \end{syntax}
%   There are alternative forms of the digit $6$. While today the usual form is
%   |stroke|, there are also historic documents showing the $6$ as two dots,
%   which you can achieve using the |dot| choice, and to support squares or
%   short dashes instead of round dots you can use the |dash| choice.
%
%   For |dot| you can also specify the radius as a dimension. The default is
%   \mydefault{six@dot}. And for |dash| you can also specify the length in
%   multiples of the used stroke thickness, an empty value is considered the
%   same as $1$. The default is \mydefault{six@dash}. The |stroke| has always
%   the full length and doesn't accept an additional specification.
%
%   A comparison of the alternate forms is shown in figure~\ref{fig:69}.
% \end{function}
%
% \begin{function}{alternate-9,9}
%   \begin{syntax}
%     \cs{alternate-9} = \marg{choice} \mysdefault{rectangle}
%   \end{syntax}
%   Another digit having alternative forms is the $9$. For this two
%   \meta{choice}s are available, |rectangle| and |triangle|. Neither of the two
%   accept additional specifications.
%
%   A comparison of the alternate forms is shown in figure~\ref{fig:69}.
% \end{function}
%
% \begin{figure}
%   \centering
%   \exampledigits{6666}%
%   \cisterciansetup{vex}%
%   \begin{tabular}{@{}*5c}
%     \texttt{stroke}       & \texttt{dot=2pt}       & \texttt{dot=4pt}
%                           & \texttt{dash=1}        & \texttt{dash=2}        \\
%     \showbb{6=stroke}     & \showbb{6=dot=2pt}     & \showbb{6=dot=4pt}
%                           & \showbb{6=dash=1}      & \showbb{6=dash=2}      \\
%     \showbb{6=stroke,hex} & \showbb{6=dot=2pt,hex} & \showbb{6=dot=4pt,hex}
%                           & \showbb{6=dash=1, hex} & \showbb{6=dash=2, hex} \\
%   \end{tabular}\hfill
%   \exampledigits{9999}%
%   \begin{tabular}{*2c@{}}
%     \texttt{rectangle}       & \texttt{triangle}       \\
%     \showbb{9=rectangle}     & \showbb{9=triangle}     \\
%     \showbb{9=rectangle,hex} & \showbb{9=triangle,hex} \\
%   \end{tabular}
%   \caption[Alternate forms of the digits $6$ and $9$]
%     {Alternate forms of the digits $6$ (left) and $9$ (right). The tight
%     bounding boxes using \texttt{bx=1,by=1} are drawn red.\label{fig:69}}
% \end{figure}
%
% \begin{function}{font,f}
%   \begin{syntax}
%     font = \marg{font setup} \mydefault{font}
%   \end{syntax}
%   Set the font which \pkg{xistercian} uses locally to evaluate the dimensions
%   given to |width|, |height|, and |stroke| (only font switches work here, you
%   can't use stuff like |\texttt|, instead use |\ttfamily|).
% \end{function}
%
% \begin{function}{redraw,r}
%   \begin{syntax}
%     redraw = \marg{feature-list} \mysdefault{size}
%     redraw += \marg{feature-list}
%     redraw
%     redraw!
%   \end{syntax}
%   Selects which font features to consider when deciding to redraw the glyphs.
%   If one of the features has changed (or the features to consider) the glyphs
%   will be redrawn. This is especially handy if you change the |font| used
%   while evaluating the given dimensions. The available features in the comma
%   separated \meta{feature-list} are |encoding|, |family|, |series|, |shape|,
%   |size|, and |color|.
%
%   If you see that a Cistercian numeral doesn't have the correct colour of the
%   surrounding text you need to add |color| to the font features
%   (\pkg{xistercian} uses some hack to get the boxed glyphs to respect the
%   surrounding colour, but that hack might fail in certain circumstances).
%
%   If you use the |+=|~variant of this option the \meta{feature-list} is added
%   to the existing features, else they overwrite any existing configuration.
%
%   Without a value the key will issue |\cistercianredrawlazy|, and if you use
%   the |!|~variant it'll use |\cistercianredraw|. This usage is only available
%   after the package was loaded.
% \end{function}
%
% \begin{function}{debug}
%   \begin{syntax}
%     debug
%   \end{syntax}
%   Only usable as a package option. If this is used a bit of debugging
%   information is printed in the log file and terminal, and some keys try to
%   evaluate the given argument on the spot instead of lazily. This might help
%   to locate faulty input.
% \end{function}
%
%
% \subsection{Bold Cistercian Numerals}
%
% While the package doesn't directly support bold glyphs, it is possible to
% utilize the hooks provided by the NFSS to pick different stroke widths
% whenever the font is changed to |bfseries| and back to |mdseries|:
%
% \begin{verbatim}
% \AddToHook{bfseries}
%   {%
%     \cisterciansetup{sv=.27ex,sdd=.27ex,sh=.15ex,sdu=.15ex,wd=.625em}%
%     \cistercianredrawlazy
%   }
% \AddToHook{mdseries}
%   {%
%     \cisterciansetup{sv=.18ex,sdd=.18ex,sh=.1ex,sdu=.1ex,wd=.5em}%
%     \cistercianredrawlazy
%   }
% This results in \cisterciannum{2} different \enquote{series}.\par
% \bfseries
% This results in \cisterciannum{2} different \enquote{series}.
% \end{verbatim}
%
% \begingroup
% This results in \cisterciannum{2} different \enquote{series}.\par
% \bfseries
% This results in \cisterciannum{2} different \enquote{series}.
% \endgroup
%
% Of course, instead of doing |\cistercianredrawlazy| in the above code, we
% could as well do the following equivalent thing and let \pkg{xistercian}
% determine when to redraw:
% \begin{verbatim}
% \AddToHook{bfseries}
%   {\cisterciansetup{sv=.27ex,sdd=.27ex,sh=.15ex,sdu=.15ex,wd=.625em}}
% \AddToHook{mdseries}
%   {\cisterciansetup{sv=.18ex,sdd=.18ex,sh=.1ex,sdu=.1ex,wd=.5em}}
% \cisterciansetup{redraw+=series}
% \end{verbatim}
%
%
% \clearpage
% \section{Implementation}
%
% First we say who we are and load the required packages
% \gobbledocstriptag
%<*pkg>
%    \begin{macrocode}
\ProvidesPackage{xistercian}[2021-12-20 v1.2 Cistercian numerals in LaTeX]
\RequirePackage{pgf,expkv-opt}
%    \end{macrocode}
%
%
% \subsection{Variables}
%
% There are variables stored as macros to be evaluated later, as well as
% register types for faster access during the glyph drawing.
%    \begin{macrocode}
\newcommand*\xister@th{.1ex}
\newcommand*\xister@tv{.18ex}
\newcommand*\xister@tdu{.1ex}
\newcommand*\xister@tdd{.18ex}
\newcommand*\xister@ts{}
\newcommand*\xister@x{.5em}
\newcommand*\xister@y{1.55ex}
\newcommand*\xister@bx{1.2}
\newcommand*\xister@by{}
\newcommand*\xister@bs{}
\newcommand*\xister@font{\normalfont}
\newcommand*\xister@five@dot{.09ex}
\newcommand*\xister@five@dash{}
\newcommand*\xister@six@dot{.09ex}
\newcommand*\xister@six@dash{}
\newcommand*\xister@share@div{/6}
\newcommand*\xister@share@mul{}
\newif\ifxister@dbg@
\newif\ifxister@five@triangle@\xister@five@triangle@true
\newif\ifxister@five@dot@
\newif\ifxister@six@dot@
\newif\ifxister@six@stroke@\xister@six@stroke@true
\newif\ifxister@nine@triangle@
\newdimen\xister@X
\newdimen\xister@Y
\newdimen\xister@XY@share
\newdimen\xister@Th
\newdimen\xister@Tv
\newdimen\xister@Tdu
\newdimen\xister@Tdd
\newdimen\xister@Ts
\newdimen\xister@TMPA
\newdimen\xister@TMPB
\newdimen\xister@Five@dot
\newdimen\xister@Six@dot
%    \end{macrocode}
%
% We also need a bunch of box registers ($37$ of them to be precise). For the
% loop we borrow an \pkg{expl3} function.
%    \begin{macrocode}
\ExplSyntaxOn
\int_step_inline:nn {9}
  {
    \expandafter\newsavebox\csname xister@digitbox@#1\endcsname
    \expandafter\newsavebox\csname xister@digitbox@#10\endcsname
    \expandafter\newsavebox\csname xister@digitbox@#100\endcsname
    \expandafter\newsavebox\csname xister@digitbox@#1000\endcsname
  }
\ExplSyntaxOff
\expandafter\newsavebox\csname xister@digitbox@0\endcsname
%    \end{macrocode}
%
% \begin{macro}[internal]{\xister@dbg@dim,\xister@dbg@float}
%   These are initially just |\@gobble|, but the |debug| option might change
%   things.
%    \begin{macrocode}
\let\xister@dbg@dim\@gobble
\let\xister@dbg@div\@gobble
\let\xister@dbg@float\@gobble
%    \end{macrocode}
% \end{macro}
%
%
% \subsection{Options}
%
% Some macros are required to parse options.
%
% \begin{macro}[internal]{\xister@if@slash}
%   We need to test whether a slash is inside an argument to parse some user
%   option. Since this isn't needed inside an inner loop we don't create an
%   optimised version and only borrow a generic \pkg{expl3} function.
%    \begin{macrocode}
\ExplSyntaxOn
\cs_new_protected:Npn \xister@if@slash #1 { \tl_if_in:nnTF {#1} { / } }
\ExplSyntaxOff
%    \end{macrocode}
% \end{macro}
%
% \begin{macro}[internal]{\xister@fraction}
%    \begin{macrocode}
\protected\long\def\xister@fraction#1/#2\xister@stop
  {%
    \edef\xister@share@mul{\unexpanded{#1}}%
    \xister@dbg@float{#1}%
    \edef\xister@share@div{/\unexpanded{#2}}%
    \xister@dbg@div{#2}%
  }
%    \end{macrocode}
% \end{macro}
%
% \begin{macro}[internal]
%   {
%     \xister@build,
%     \xister@build@h,\xister@build@horizontal,
%     \xister@build@v,\xister@build@vertical
%   }
%   First we set the length registers to the correct size, then we draw the
%   glyphs batch wise. We have to flip the $x$-direction for the tenths and
%   thousands, and the $y$-direction for the hundreds and thousands. The macro
%   |\xister@sgn| is used to set the baseline in the correct direction. We
%   locally disable |\pgfsetcolor| to allow building colour-unsafe boxes. This
%   only works as long as \pkg{luacolor} (or something equivalent) isn't used,
%   so consider this a crude hack.
%    \begin{macrocode}
\protected\def\xister@build@vertical
  {%
    \let\xister@pgfsetcolor\pgfsetcolor
    \let\pgfsetcolor\@gobble
    \xister@setlengths
    \def\xister@sgn{-}%
    \xister@drawzero@vertical
    \xister@drawdigits@vertical{}{}{}%
    \xister@X=-\xister@X
    \xister@Y=-\xister@Y
    \def\xister@sgn{+}%
    \xister@drawdigits@vertical{000}{-}{-}%
    \xister@TMPA=\xister@Tdd
    \xister@Tdd=\xister@Tdu
    \xister@Tdu=\xister@TMPA
    \xister@X=-\xister@X
    \xister@drawdigits@vertical{00}{-}{}%
    \xister@X=-\xister@X
    \xister@Y=-\xister@Y
    \def\xister@sgn{-}%
    \xister@drawdigits@vertical{0}{}{-}%
    \let\pgfsetcolor\xister@pgfsetcolor
  }
\protected\def\xister@build@horizontal
  {%
    \let\xister@pgfsetcolor\pgfsetcolor
    \let\pgfsetcolor\@gobble
    \xister@setlengths
    \def\xister@sgn{-}%
    \xister@drawzero@horizontal
    \xister@drawdigits@horizontal{}{}{}%
    \xister@X=-\xister@X
    \xister@Y=-\xister@Y
    \def\xister@sgn{+}%
    \xister@drawdigits@horizontal{000}{-}{-}%
    \xister@TMPA=\xister@Tdd
    \xister@Tdd=\xister@Tdu
    \xister@Tdu=\xister@TMPA
    \xister@X=-\xister@X
    \xister@drawdigits@horizontal{0}{-}{}%
    \xister@X=-\xister@X
    \xister@Y=-\xister@Y
    \def\xister@sgn{-}%
    \xister@drawdigits@horizontal{00}{}{-}%
    \let\pgfsetcolor\xister@pgfsetcolor
  }
\let\xister@build\xister@build@vertical
\let\xister@build@v\xister@build@vertical
\let\xister@build@h\xister@build@horizontal
%    \end{macrocode}
% \end{macro}
%
% \begin{macro}[internal]{\xister@fontfeatures,\xister@add@fontfeature}
%   This macro should store the different font features which should be kept
%   track of. It will be redefined programmatically but the default is quite
%   easy, just track the font size.
%    \begin{macrocode}
\protected\long\def\xister@add@fontfeature#1%
  {%
    \begingroup\expandafter\expandafter\expandafter\endgroup
    \expandafter\ifx
        \csname xister@add@fontfeature@\detokenize{#1}\endcsname
        \relax
      \PackageError{xistercian}{Unsupported font feature `\detokenize{#1}'.}{}%
    \else
      \csname xister@add@fontfeature@\detokenize{#1}\endcsname
    \fi
  }
\def\xister@fontfeatures{\unexpanded\expandafter{\f@size}/}
%    \end{macrocode}
%   We know that the colour hack breaks if \pkg{luacolor} is loaded, so if that
%   is found to be the case we add the |color| feature to the list of things we
%   need to track. But we only want to add it if it's not yet in the list, so we
%   run a check we borrow from \pkg{expl3}.
%    \begin{macrocode}
\ExplSyntaxOn
\AddToHook{begindocument/before}
  {
    \@ifpackageloaded { luacolor }
      {
        \exp_args:No \str_if_in:nnF \xister@fontfeatures { color }
          { \xister@add@fontfeature{color} }
      }
      {}
  }
\ExplSyntaxOff
%    \end{macrocode}
% \end{macro}
%
% \begin{macro}[internal]
%   {
%     \xister@add@fontfeature@encoding,
%     \xister@add@fontfeature@family,
%     \xister@add@fontfeature@series,
%     \xister@add@fontfeature@shape,
%     \xister@add@fontfeature@size,
%   }
%   These macros all just add a specific macro to the list of things contained
%   in |\xister@fontfeatures|, which determines what to consider deciding to
%   redraw.
%    \begin{macrocode}
\def\xister@tmp#1#2%
  {%
    \protected\expandafter\def\csname xister@add@fontfeature@#2\endcsname
      {%
        \edef\xister@fontfeatures
          {%
            \unexpanded\expandafter
              {\xister@fontfeatures\unexpanded\expandafter{#1}}/%
          }%
      }%
  }
\expandafter\xister@tmp\csname\@backslashchar color@.\endcsname{color}
\xister@tmp\f@encoding{encoding}
\xister@tmp\f@family  {family}
\xister@tmp\f@series  {series}
\xister@tmp\f@shape   {shape}
\xister@tmp\f@size    {size}
%    \end{macrocode}
% \end{macro}
%
% Now we define the keys
%    \begin{macrocode}
\ekvifdefinedset{xister}
  {\PackageError{xistercian}{keyval conflict detected. Aborting}{}\endinput}{}
\def\xister@tmp#1#2#3%
  {%
    \protected\long\ekvdef{xister}{#1}{#3}%
    \ekvletkv{xister}{#2}{xister}{#1}%
  }
\xister@tmp{wd}{width}{\edef\xister@x{\unexpanded{#1}}\xister@dbg@dim{#1}}
\xister@tmp{ht}{height}{\edef\xister@y{\unexpanded{#1}}\xister@dbg@dim{#1}}
\xister@tmp{bx}{bound-x}{\edef\xister@bx{\unexpanded{#1}}\xister@dbg@float{#1}}
\xister@tmp{by}{bound-y}{\edef\xister@by{\unexpanded{#1}}\xister@dbg@float{#1}}
\xister@tmp{bs}{baseline}{\edef\xister@bs{\unexpanded{#1}}\xister@dbg@float{#1}}
\xister@tmp{sv}{stroke-v}{\edef\xister@tv{\unexpanded{#1}}\xister@dbg@dim{#1}}
\xister@tmp{sh}{stroke-h}{\edef\xister@th{\unexpanded{#1}}\xister@dbg@dim{#1}}
\xister@tmp{sdu}{stroke-du}
  {\edef\xister@tdu{\unexpanded{#1}}\xister@dbg@dim{#1}}
\xister@tmp{sdd}{stroke-dd}
  {\edef\xister@tdd{\unexpanded{#1}}\xister@dbg@dim{#1}}
\xister@tmp{ss}{stroke-s}
  {%
    \edef\xister@ts{\unexpanded{#1}}%
    \unless\ifx\xister@ts\@empty\xister@dbg@dim{#1}\fi
  }
\xister@tmp{s}{strokes}
  {%
    \edef\xister@tv{\unexpanded{#1}}%
    \let\xister@th\xister@tv
    \let\xister@tdu\xister@tv
    \let\xister@tdd\xister@tv
    \let\xister@ts\xister@tv
    \xister@dbg@dim{#1}%
  }
\xister@tmp{f}{font}{\edef\xister@font{\unexpanded{#1}}}
\xister@tmp{o}{orientation}
  {%
    \begingroup\expandafter\expandafter\expandafter\endgroup
    \expandafter\ifx\csname xister@build@#1\endcsname\relax
      \PackageError{xister}%
        {Unaccepted orientation. Choices are horizontal/h or vertical/v}{}%
    \else
      \expandafter\let\expandafter\xister@build\csname xister@build@#1\endcsname
    \fi
  }
\xister@tmp{fr}{fraction}
  {%
    \xister@if@slash{#1}%
      {\xister@fraction#1\xister@stop}%
      {%
        \let\xister@share@div\@empty
        \edef\xister@share@mul{\unexpanded{#1}}%
        \xister@dbg@float{#1}%
      }%
  }
\xister@tmp{r}{redraw}
  {\let\xister@fontfeatures\@empty\ekvcsvloop\xister@add@fontfeature{#1}}
\xister@tmp{r+}{redraw+}{\ekvcsvloop\xister@add@fontfeature{#1}}
\ekvletkv{xister}{r +}{xister}{r+}
\ekvletkv{xister}{redraw +}{xister}{redraw+}
\protected\long\ekvdef{xister/5}{dot}
  {%
    \xister@five@triangle@false
    \xister@five@dot@true
    \edef\xister@five@dot{\unexpanded{#1}}%
    \xister@dbg@dim{#1}%
  }
\protected\long\ekvdef{xister/5}{dash}
  {%
    \xister@five@triangle@false
    \xister@five@dot@false
    \edef\xister@five@dash{\unexpanded{#1}}%
    \xister@dbg@float{#1}%
  }
\protected\ekvdefNoVal{xister/5}{triangle}{\xister@five@triangle@true}
\protected\ekvdefNoVal{xister/5}{dot}
  {%
    \xister@five@triangle@false
    \xister@five@dot@true
  }
\protected\ekvdefNoVal{xister/5}{dash}
  {%
    \xister@five@triangle@false
    \xister@five@dot@false
  }
\protected\ekvsetdef\xister@five@set{xister/5}
\ekvlet{xister}{5}\xister@five@set
\ekvlet{xister}{alternate-5}\xister@five@set
\let\xister@five@set\xister@undefined
\protected\long\ekvdef{xister/6}{dot}
  {%
    \xister@six@stroke@false
    \xister@six@dot@true
    \edef\xister@six@dot{\unexpanded{#1}}%
    \xister@dbg@dim{#1}%
  }
\protected\long\ekvdef{xister/6}{dash}
  {%
    \xister@six@stroke@false
    \xister@six@dot@false
    \edef\xister@six@dash{\unexpanded{#1}}%
    \xister@dbg@float{#1}%
  }
\protected\ekvdefNoVal{xister/6}{stroke}{\xister@six@stroke@true}
\protected\ekvdefNoVal{xister/6}{dot}
  {%
    \xister@six@stroke@false
    \xister@six@dot@true
  }
\protected\ekvdefNoVal{xister/6}{dash}
  {%
    \xister@six@stroke@false
    \xister@six@dot@false
  }
\protected\ekvsetdef\xister@six@set{xister/6}
\ekvlet{xister}{6}\xister@six@set
\ekvlet{xister}{alternate-6}\xister@six@set
\let\xister@six@set\xister@undefined
\protected\ekvdefNoVal{xister/9}{rectangle}{\xister@nine@triangle@false}
\protected\ekvdefNoVal{xister/9}{triangle}{\xister@nine@triangle@true}
\protected\ekvsetdef\xister@nine@set{xister/9}
\ekvlet{xister}{9}\xister@nine@set
\ekvlet{xister}{alternate-9}\xister@nine@set
\protected\ekvdefNoVal{xister}{horizontal}
  {%
    \let\xister@build\xister@build@horizontal
    \def\xister@x{.775em}%
    \def\xister@y{1.13ex}%
  }
\protected\ekvdefNoVal{xister}{vertical}
  {%
    \let\xister@build\xister@build@vertical
    \def\xister@x{.5em}%
    \def\xister@y{1.55ex}%
  }
%    \end{macrocode}
% The |debug| option redefines the debug macros to the correct definition.
%    \begin{macrocode}
\protected\ekvdefNoVal{xister}{debug}
  {%
    \xister@dbg@true
    \protected\long\def\xister@dbg@dim##1%
      {%
        \typeout
          {%
            Package xistercian Debug: Trying to use dimension
            `\unexpanded{##1}'.%
          }%
        \xister@TMPA=\dimexpr##1\relax
        \typeout{Package xistercian Debug: Done.}%
      }%
    \protected\long\def\xister@dbg@float##1%
      {%
        \typeout
          {Package xistercian Debug: Trying to use float `\unexpanded{##1}'.}%
        \xister@TMPA=##1\z@
        \typeout{Package xistercian Debug: Done.}%
      }%
    \protected\long\def\xister@dbg@div##1%
      {%
        \typeout
          {Package xistercian Debug: Trying to use divisor `\unexpanded{##1}'.}%
        \xister@TMPA=\dimexpr\z@/##1\relax
        \typeout{Package xistercian Debug: Done.}%
      }%
  }
%    \end{macrocode}
% and parse the package options (and since the used names |width|, |height|, and
% |debug| are quite frequent we don't look at the global options).
%    \begin{macrocode}
\ekvoProcessLocalOptions{xister}
%    \end{macrocode}
%
% After the package was loaded we add the two |redraw| variants without a value.
%    \begin{macrocode}
\protected\ekvdefNoVal{xister}{r!}{\cistercianredraw}
\protected\ekvdefNoVal{xister}{r} {\cistercianredrawlazy}
\ekvletkvNoVal{xister}{redraw} {xister}{r}
\ekvletkvNoVal{xister}{redraw!}{xister}{r!}
%    \end{macrocode}
%
% The |debug| option is only available as a package option, we let it throw an
% error:
%    \begin{macrocode}
\ekvdefNoVal{xister}{debug}
  {\PackageError{xister}{`debug' is only available as a load time option}{}}
%    \end{macrocode}
%
%
% \subsection{User macros}
%
% \begin{macro}{\cisterciansetup}
%   Just as a simple way to pick options later on.
%    \begin{macrocode}
\protected\ekvsetdef\cisterciansetup{xister}
%    \end{macrocode}
% \end{macro}
%
% \begin{macro}{\cistercianstyle}
% \begin{macro}[internal]{\xister@style}
%   We allow users to define a style they can call with a single key name inside
%   |\cisterciansetup|. To achieve that we just define a |NoVal| key that'll
%   call a nested |\cisterciansetup|, or a normal key if their style needs to
%   take a value.
%    \begin{macrocode}
\protected\def\cistercianstyle
  {%
    \@ifstar
      {\xister@style\ekvifdefined{\long\ekvdef}}%
      {\xister@style\ekvifdefinedNoVal\ekvdefNoVal}%
  }
\protected\long\def\xister@style#1#2#3#4%
  {%
    #1{xister}{#3}%
      {\PackageError{xistercian}{Key `#3' already defined}{}}%
      {\protected#2{xister}{#3}{\cisterciansetup{#4}}}%
  }
%    \end{macrocode}
% \end{macro}
% \end{macro}
%
% \begin{macro}{\cistercian}
% \begin{macro}[internal]{\@cistercian}
%   The macro is equivalent to other \LaTeX\ counter formatting macros, so we
%   let it build the counter name and forward that to the canonically named
%   auxiliary macro |\@cistercian|.
%    \begin{macrocode}
\newcommand\cistercian[1]{\expandafter\@cistercian\csname c@#1\endcsname}
%    \end{macrocode}
%   The internal just gets the current value of a counter and forwards it.
%    \begin{macrocode}
\newcommand\@cistercian[1]{\expandafter\cisterciannum\expandafter{\the#1}}
%    \end{macrocode}
% \end{macro}
% \end{macro}
%
%
% \begin{macro}{\cisterciannum}
% \begin{macro}[internal]{\cisterciannum@pdf}
%    \begin{macrocode}
\edef\cisterciannum
  {%
    \unexpanded{\xister@texorpdf}%
    \unexpanded\expandafter
      {\csname cisterciannum \endcsname\cisterciannum@pdf}%
  }
\protected\expandafter\def\csname cisterciannum \endcsname#1{\xister@a#1@}
\def\cisterciannum@pdf#1{CISTER#1}
%    \end{macrocode}
% \end{macro}
% \end{macro}
%
% \begin{macro}{\cisterciannumE}
%    \begin{macrocode}
\newcommand\cisterciannumE[1]
  {\expandafter\cisterciannum\expandafter{\the\numexpr#1\relax}}
%    \end{macrocode}
% \end{macro}
%
% \begin{macro}{\cistercianredraw,\cistercianredrawlazy}
%    \begin{macrocode}
\protected\def\cistercianredraw
  {%
    \def\xister@last@font{\xister@last@font}%
    \xister@ensure@current
  }
\protected\def\cistercianredrawlazy{\def\xister@last@font{\xister@last@font}}
%    \end{macrocode}
% \end{macro}
%
%
% \subsection{Parsing}
%
% \subsubsection{Small Auxiliaries}
%
% \begin{macro}[internal]
%   {\xister@gobbletoat,\xister@gobbletozero,\xister@gobblecopy,\xister@done}
%   These are just small functions gobbling some tokens.
%    \begin{macrocode}
\def\xister@gobbletoat#1@{}
\def\xister@gobbletozero#10{}
\def\xister@gobblecopy0\rlap#1{}
\def\xister@done#1\xister@symbols{}
%    \end{macrocode}
% \end{macro}
%
% \begin{macro}[internal]{\xister@texorpdf}
%    \begin{macrocode}
\let\xister@texorpdf\@firstoftwo
\AddToHook{package/hyperref/after}
  {%
    \ifdefined\texorpdfstring
      \def\xister@texorpdf{\texorpdfstring}%
    \fi
  }
\AddToHook{begindocument/before}
  {%
    \ifdefined\texorpdfstring
      \def\xister@texorpdf{\texorpdfstring}%
    \fi
  }
%    \end{macrocode}
% \end{macro}
%
%
% \subsubsection{Input Parsing}
%
% \begin{macro}[internal]{\xister@a,\xister@b,\xister@c,\xister@d}
%   First we have to make sure that we're in horizontal mode, else the |\rlap|s
%   used to insert the glyphs will create havoc. We also need to ensure that the
%   boxes contain the correct glyphs according to the current size and colour.
%   After that we check whether we have to step over a minus sign and go to the
%   next step.
%    \begin{macrocode}
\def\xister@a#1%
  {%
    \leavevmode
    \xister@ensure@current
    \ifx-#1-\expandafter\xister@b
    \else\expandafter\xister@b\expandafter#1\fi
  }
%    \end{macrocode}
%   Next we check whether the remaining number is empty. If that is the case we
%   use |\the\numexpr\relax| to throw a missing number error and produce just a
%   zero stem. Else we do a dry run over the input number to see how many
%   leading zeros we must pad to get blocks of four. The two blocks of |@|s
%   serve as end markers since each step of both \cs{xister@c} grabs eight and
%   each of \cs{xister@symbols} will grab four digits. The \cs{xister@c} is
%   used to put enough zeros in front of our number such that it consists of
%   blocks of four digits.
%    \begin{macrocode}
\def\xister@b#1@%
  {%
    \if\relax\detokenize{#1}\relax
      \expandafter\xister@gobbletozero\the\numexpr\relax
      \xister@symbols0000@@@@%
      \expandafter\@gobbletwo
    \fi
    \@firstofone{\xister@c#1@@@@@@@@\relax#1@@@@}%
  }
\def\xister@c#1#2#3#4#5#6#7#8%
  {%
    \xister@gobbletoat
      #1\xister@d{}%
      #2\xister@d{000}%
      #3\xister@d{00}%
      #4\xister@d0%
      #5\xister@d{}%
      #6\xister@d{000}%
      #7\xister@d{00}%
      #8\xister@d0%
      @\xister@c
  }
\def\xister@d#1#2@\xister@c#3\relax{\xister@symbols#1}
%    \end{macrocode}
% \end{macro}
%
% \begin{macro}[internal]{\xister@symbols}
%   When this loop is done all arguments will be |@|, but we only check the
%   first here. We generate the symbols by overlaying up to five boxes, one for
%   each decade plus the zero stem. Each digit might be skipped if it is $0$.
%   Next grab the next four digits.
%    \begin{macrocode}
\def\xister@symbols#1%
  {%
    \protected\def\xister@symbols##1##2##3##4%
      {%
        \xister@gobbletoat##1\xister@done @%
        \xister@gobbletozero##1\xister@gobblecopy0%
          \rlap{\copy\csname xister@digitbox@##1000\endcsname}%
        \xister@gobbletozero##2\xister@gobblecopy0%
          \rlap{\copy\csname xister@digitbox@##200\endcsname}%
        \xister@gobbletozero##3\xister@gobblecopy0%
          \rlap{\copy\csname xister@digitbox@##30\endcsname}%
        \xister@gobbletozero##4\xister@gobblecopy0%
          \rlap{\copy\csname xister@digitbox@##4\endcsname}%
        \copy#1%
        \xister@symbols
      }%
  }
\expandafter\xister@symbols\csname xister@digitbox@0\endcsname
%    \end{macrocode}
% \end{macro}
%
%
% \subsection{Drawing the Digits}
%
% \begin{macro}[internal]{\xister@clip@v,\xister@clip@h}
%   Some of the digit glyphs need a bit of clipping to look nice. Let's define a
%   small auxiliary to select a clipping region.
%    \begin{macrocode}
\newcommand*\xister@clip@v[1]
  {%
    \pgfpathrectanglecorners
      {\pgfpoint{\dimexpr\ifdim\xister@X<\z@-\fi\xister@Ts\relax}{#1}}%
      {\pgfpoint\xister@X\xister@Y}%
    \pgfusepath{clip}%
  }
\newcommand*\xister@clip@h[1]
  {%
    \pgfpathrectanglecorners
      {\pgfpoint{#1}{-\dimexpr\ifdim\xister@Y<\z@-\fi\xister@Ts\relax}}%
      {\pgfpoint{-\xister@X}\xister@Y}%
    \pgfusepath{clip}%
  }
%    \end{macrocode}
% \end{macro}
%
% \begin{macro}[internal]
%   {\xister@drawdigits@vertical,\xister@drawdigits@horizontal}
%   The digit glyphs. Since each numeral is build from the zero stem and some
%   appendix to it this only specifies the form of the appendices. The zero stem
%   is defined a bit down the road. |#1| will determine the decade, and |#2|
%   will be either~|-| or empty.
%    \begin{macrocode}
\newcommand\xister@drawdigits@vertical[3]
  {%
    \xister@XY@share=\dimexpr\xister@share@mul\xister@Y\xister@share@div\relax
    \xister@drawdigit{1#1}%
      {%
        \pgfsetlinewidth{2\xister@Th}%
        \xister@TMPA=\dimexpr\xister@Y\xister@sgn\xister@Th\relax
        \pgfpathmoveto{\pgfpoint\z@\xister@TMPA}%
        \pgfpathlineto{\pgfpoint\xister@X\xister@TMPA}%
      }%
    \xister@drawdigit{2#1}%
      {%
        \pgfsetlinewidth{2\xister@Th}%
        \xister@TMPA=\dimexpr\xister@XY@share+#2\xister@Th\relax
        \pgfpathmoveto{\pgfpoint\z@\xister@TMPA}%
        \pgfpathlineto{\pgfpoint\xister@X\xister@TMPA}%
      }%
    \xister@drawdigit{3#1}%
      {%
        \xister@TMPB=#3\dimexpr\xister@Ts-\xister@Tdd\relax
        \xister@clip@v\xister@XY@share
        \pgfsetlinewidth{2\xister@Tdd}%
        \pgfpathmoveto{\pgfpoint\xister@TMPB\xister@XY@share}%
        \pgfpathlineto{\pgfpoint\xister@TMPB\xister@Y}%
        \pgfpathlineto{\pgfpoint\xister@X\xister@XY@share}%
      }%
    \xister@drawdigit{4#1}%
      {%
        \xister@clip@v\z@
        \xister@TMPB=#3\dimexpr\xister@Ts-\xister@Tdu\relax
        \pgfsetlinewidth{2\xister@Tdu}%
        \pgfpathmoveto{\pgfpoint\xister@TMPB\xister@Y}%
        \pgfpathlineto{\pgfpoint\xister@TMPB\xister@XY@share}%
        \pgfpathlineto{\pgfpoint\xister@X\xister@Y}%
      }%
    \xister@drawdigit{5#1}%
      {%
        \ifxister@five@triangle@
          \xister@clip@v\z@
          \xister@TMPB=%
            #3\dimexpr\xister@Ts-\xister@Tdu\relax
          \pgfsetlinewidth{2\xister@Tdu}%
          \pgfpathmoveto{\pgfpoint\xister@TMPB\xister@Y}%
          \pgfpathlineto{\pgfpoint\xister@TMPB\xister@XY@share}%
          \pgfpathlineto{\pgfpoint\xister@X\xister@Y}%
          \pgfusepath{stroke,clip}%
          \pgfsetlinewidth{2\xister@Th}%
          \xister@TMPA=\dimexpr\xister@Y\xister@sgn\xister@Th\relax
          \pgfpathmoveto{\pgfpoint\z@\xister@TMPA}%
          \pgfpathlineto{\pgfpoint\xister@X\xister@TMPA}%
        \else\ifxister@five@dot@
          \pgfpathcircle
            {\pgfpoint{\dimexpr\xister@X-#3\xister@Five@dot\relax}%
                      {\dimexpr\xister@Y-#2\xister@Five@dot\relax}}%
            {\xister@Five@dot}%
          \pgfusepath{fill}%
        \else
          \xister@TMPB=\dimexpr\xister@X-#3\xister@Tv\relax
          \pgfsetlinewidth{2\xister@Tv}%
          \pgfpathmoveto{\pgfpoint\xister@TMPB\xister@Y}%
          \pgfpathlineto
            {%
              \pgfpoint
                \xister@TMPB
                {\dimexpr\xister@Y-#2\xister@five@dash\xister@Tv*2\relax}%
            }%
        \fi\fi
      }%
    \xister@drawdigit{6#1}%
      {%
        \pgfsetlinewidth{2\xister@Tv}%
        \xister@TMPB=\dimexpr\xister@X-#3\xister@Tv\relax
        \pgfpathmoveto{\pgfpoint\xister@TMPB\xister@Y}%
        \pgfpathlineto{\pgfpoint\xister@TMPB\xister@XY@share}%
      }%
    \xister@superimpose{7#1}%
      {%
        \rlap{\copy\csname xister@digitbox@1#1\endcsname}%
        \copy\csname xister@digitbox@6#1\endcsname
      }%
    \xister@superimpose{8#1}%
      {%
        \rlap{\copy\csname xister@digitbox@2#1\endcsname}%
        \copy\csname xister@digitbox@6#1\endcsname
      }%
    \ifxister@nine@triangle@
      \xister@drawdigit{9#1}%
        {%
          \xister@TMPA=.5\dimexpr\xister@XY@share+\xister@Y\relax
          \xister@TMPB=#3\dimexpr\xister@Ts-\xister@Tdd\relax
          \xister@clip@v\z@
          \pgfsetlinewidth{2\xister@Tdd}%
          \pgfpathmoveto{\pgfpoint\xister@TMPB\xister@XY@share}%
          \pgfpathlineto{\pgfpoint\xister@TMPB\xister@Y}%
          \pgfpathlineto{\pgfpoint\xister@X\xister@TMPA}%
          \pgfusepath{stroke}%
          \xister@TMPB=#3\dimexpr\xister@Ts-\xister@Tdu\relax
          \pgfsetlinewidth{2\xister@Tdu}%
          \pgfpathmoveto{\pgfpoint\xister@TMPB\xister@Y}%
          \pgfpathlineto{\pgfpoint\xister@TMPB\xister@XY@share}%
          \pgfpathlineto{\pgfpoint\xister@X\xister@TMPA}%
        }%
    \else
      \xister@superimpose{9#1}%
        {%
          \rlap{\copy\csname xister@digitbox@1#1\endcsname}%
          \rlap{\copy\csname xister@digitbox@2#1\endcsname}%
          \copy\csname xister@digitbox@6#1\endcsname
        }%
    \fi
    \unless\ifxister@six@stroke@
      \xister@drawdigit{6#1}%
        {%
          \ifxister@six@dot@
            \xister@TMPA=\dimexpr\xister@X-#3\xister@Six@dot\relax
            \pgfpathcircle
              {\pgfpoint\xister@TMPA
                        {\dimexpr\xister@Y-#2\xister@Six@dot\relax}}%
              {\xister@Six@dot}%
            \pgfpathcircle
              {\pgfpoint\xister@TMPA
                        {\dimexpr\xister@XY@share+#2\xister@Six@dot\relax}}%
              {\xister@Six@dot}%
            \pgfusepath{fill}%
          \else
            \xister@TMPB=\dimexpr\xister@X-#3\xister@Tv\relax
            \xister@TMPA=2\dimexpr#2\xister@six@dash\xister@Tv\relax
            \pgfsetlinewidth{2\xister@Tv}%
            \pgfpathmoveto{\pgfpoint\xister@TMPB\xister@Y}%
            \pgfpathlineto
              {\pgfpoint\xister@TMPB
                        {\dimexpr\xister@Y-\xister@TMPA\relax}}%
            \pgfpathmoveto{\pgfpoint\xister@TMPB\xister@XY@share}%
            \pgfpathlineto
              {\pgfpoint\xister@TMPB
                        {\dimexpr\xister@XY@share+\xister@TMPA\relax}}%
          \fi
        }%
    \fi
  }
\newcommand\xister@drawdigits@horizontal[3]
  {%
    \xister@XY@share=-\dimexpr\xister@share@mul\xister@X\xister@share@div\relax
    \xister@drawdigit{1#1}%
      {%
        \pgfsetlinewidth{2\xister@Tv}%
        \xister@TMPA=\dimexpr#3\xister@Tv-\xister@X\relax
        \pgfpathmoveto{\pgfpoint\xister@TMPA\z@}%
        \pgfpathlineto{\pgfpoint\xister@TMPA\xister@Y}%
      }%
    \xister@drawdigit{2#1}%
      {%
        \pgfsetlinewidth{2\xister@Tv}%
        \xister@TMPA=\dimexpr\xister@XY@share-#3\xister@Tv\relax
        \pgfpathmoveto{\pgfpoint\xister@TMPA\z@}%
        \pgfpathlineto{\pgfpoint\xister@TMPA\xister@Y}%
      }%
    \xister@drawdigit{3#1}%
      {%
        \xister@TMPB=#2\dimexpr\xister@Ts-\xister@Tdu\relax
        \xister@clip@h\xister@XY@share
        \pgfsetlinewidth{2\xister@Tdu}%
        \pgfpathmoveto{\pgfpoint\xister@XY@share\xister@TMPB}%
        \pgfpathlineto{\pgfpoint{-\xister@X}\xister@TMPB}%
        \pgfpathlineto{\pgfpoint\xister@XY@share\xister@Y}%
      }%
    \xister@drawdigit{4#1}%
      {%
        \xister@clip@h\z@
        \xister@TMPB=#2\dimexpr\xister@Ts-\xister@Tdd\relax
        \pgfsetlinewidth{2\xister@Tdd}%
        \pgfpathmoveto{\pgfpoint\xister@X\xister@TMPB}%
        \pgfpathlineto{\pgfpoint\xister@XY@share\xister@TMPB}%
        \pgfpathlineto{\pgfpoint{-\xister@X}\xister@Y}%
      }%
    \xister@drawdigit{5#1}%
      {%
        \ifxister@five@triangle@
          \xister@clip@h\z@
          \xister@TMPB=#2\dimexpr\xister@Ts-\xister@Tdd\relax
          \pgfsetlinewidth{2\xister@Tdd}%
          \pgfpathmoveto{\pgfpoint{-\xister@X}\xister@TMPB}%
          \pgfpathlineto{\pgfpoint\xister@XY@share\xister@TMPB}%
          \pgfpathlineto{\pgfpoint{-\xister@X}\xister@Y}%
          \pgfusepath{stroke,clip}%
          \pgfsetlinewidth{2\xister@Tv}%
          \xister@TMPA=\dimexpr#3\xister@Tv-\xister@X\relax
          \pgfpathmoveto{\pgfpoint\xister@TMPA\z@}%
          \pgfpathlineto{\pgfpoint\xister@TMPA\xister@Y}%
        \else\ifxister@five@dot@
          \pgfpathcircle
            {\pgfpoint{\dimexpr#3\xister@Five@dot-\xister@X\relax}%
                      {\dimexpr\xister@Y-#2\xister@Five@dot\relax}}%
            {\xister@Five@dot}%
          \pgfusepath{fill}%
        \else
          \pgfsetlinewidth{2\xister@Th}%
          \xister@TMPB=\dimexpr\xister@Y-#2\xister@Th\relax
          \pgfpathmoveto{\pgfpoint{-\xister@X}\xister@TMPB}%
          \pgfpathlineto
            {%
              \pgfpoint
                {\dimexpr#3\xister@five@dash\xister@Th*2-\xister@X\relax}%
                \xister@TMPB
            }%
        \fi\fi
      }%
    \xister@drawdigit{6#1}%
      {%
        \pgfsetlinewidth{2\xister@Th}%
        \xister@TMPB=\dimexpr\xister@Y-#2\xister@Th\relax
        \pgfpathmoveto{\pgfpoint{-\xister@X}\xister@TMPB}%
        \pgfpathlineto{\pgfpoint\xister@XY@share\xister@TMPB}%
      }%
    \xister@superimpose{7#1}%
      {%
        \rlap{\copy\csname xister@digitbox@1#1\endcsname}%
        \copy\csname xister@digitbox@6#1\endcsname
      }%
    \xister@superimpose{8#1}%
      {%
        \rlap{\copy\csname xister@digitbox@2#1\endcsname}%
        \copy\csname xister@digitbox@6#1\endcsname
      }%
    \ifxister@nine@triangle@
      \xister@drawdigit{9#1}%
        {%
          \xister@TMPA=.5\dimexpr\xister@XY@share-\xister@X\relax
          \xister@TMPB=#2\dimexpr\xister@Ts-\xister@Tdu\relax
          \xister@clip@h\z@
          \pgfsetlinewidth{2\xister@Tdu}%
          \pgfpathmoveto{\pgfpoint\xister@XY@share\xister@TMPB}%
          \pgfpathlineto{\pgfpoint{-\xister@X}\xister@TMPB}%
          \pgfpathlineto{\pgfpoint\xister@TMPA\xister@Y}%
          \pgfusepath{stroke}%
          \xister@TMPB=#2\dimexpr\xister@Ts-\xister@Tdd\relax
          \pgfsetlinewidth{2\xister@Tdd}%
          \pgfpathmoveto{\pgfpoint\xister@X\xister@TMPB}%
          \pgfpathlineto{\pgfpoint\xister@XY@share\xister@TMPB}%
          \pgfpathlineto{\pgfpoint\xister@TMPA\xister@Y}%
        }%
    \else
      \xister@superimpose{9#1}%
        {%
          \rlap{\copy\csname xister@digitbox@1#1\endcsname}%
          \rlap{\copy\csname xister@digitbox@2#1\endcsname}%
          \copy\csname xister@digitbox@6#1\endcsname
        }%
    \fi
    \unless\ifxister@six@stroke@
      \xister@drawdigit{6#1}%
        {%
          \ifxister@six@dot@
            \xister@TMPA=\dimexpr\xister@Y-#2\xister@Six@dot\relax
            \pgfpathcircle
              {\pgfpoint{\dimexpr#3\xister@Six@dot-\xister@X\relax}%
                        \xister@TMPA}%
              {\xister@Six@dot}%
            \pgfpathcircle
              {\pgfpoint{\dimexpr\xister@XY@share-#3\xister@Six@dot\relax}%
                        \xister@TMPA}%
              {\xister@Six@dot}%
            \pgfusepath{fill}%
          \else
            \pgfsetlinewidth{2\xister@Th}%
            \xister@TMPB=\dimexpr\xister@Y-#2\xister@Th\relax
            \xister@TMPA=2\dimexpr#3\xister@six@dash\xister@Th\relax
            \pgfpathmoveto{\pgfpoint{-\xister@X}\xister@TMPB}%
            \pgfpathlineto
              {\pgfpoint{\dimexpr\xister@TMPA-\xister@X\relax}%
                        \xister@TMPB}%
            \pgfpathmoveto{\pgfpoint\xister@XY@share\xister@TMPB}%
            \pgfpathlineto
              {\pgfpoint{\dimexpr\xister@XY@share-\xister@TMPA\relax}%
                        \xister@TMPB}%
          \fi
        }%
    \fi
  }
%    \end{macrocode}
%   Each digit has the same bounding box and baseline which is specified here.
%   |#1| will be the name, |#2| the digit specific path.
%    \begin{macrocode}
\newcommand\xister@drawdigit[2]
  {%
    \expandafter\setbox\csname xister@digitbox@#1\endcsname=\hbox
      {%
        \begin{pgfpicture}%
          \pgfpathrectanglecorners
            {\pgfpoint{-\xister@bx\xister@X}{-\xister@by\xister@Y}}%
            {\pgfpoint{\xister@bx\xister@X}{\xister@by\xister@Y}}%
          \pgfusepath{use as bounding box}%
          #2%
          \pgfusepath{stroke}%
          \pgfsetbaseline{\xister@sgn\xister@bs\xister@Y}%
        \end{pgfpicture}%
      }%
  }
%    \end{macrocode}
% \end{macro}
%
%
% \begin{macro}[internal]
%   {\xister@drawzero@horizontal,\xister@drawzero@vertical}
%    \begin{macrocode}
\newcommand*\xister@drawzero@vertical
  {%
    \xister@drawdigit{0}%
      {%
        \pgfsetlinewidth{2\xister@Ts}%
        \pgfpathmoveto{\pgfpoint\z@{-\xister@Y}}%
        \pgfpathlineto{\pgfpoint\z@\xister@Y}%
      }%
  }
\newcommand*\xister@drawzero@horizontal
  {%
    \xister@drawdigit{0}%
      {%
        \pgfsetlinewidth{2\xister@Ts}%
        \pgfpathmoveto{\pgfpoint{-\xister@X}\z@}%
        \pgfpathlineto{\pgfpoint\xister@X\z@}%
      }%
  }
%    \end{macrocode}
% \end{macro}
%
% \begin{macro}[internal]{\xister@superimpose}
%   Some digits can be build from a few of the other ones. This is faster than
%   drawing everything again, and this macro eases the process a bit.
%    \begin{macrocode}
\newcommand\xister@superimpose[2]
  {\expandafter\setbox\csname xister@digitbox@#1\endcsname=\hbox{#2}}
%    \end{macrocode}
% \end{macro}
%
%
% \subsection{Rebuild the Glyphs}
%
% \begin{macro}[internal]{\xister@setlengths}
%   Lengths are set depending on the current font size and the sizes of
%   |\xister@font|. To keep the font changes local they are kept inside a group
%   and a chain of |\expandafter|s is used to evaluate the user specified sizes
%   while the |\xister@font| is still active.
%    \begin{macrocode}
\protected\def\xister@setlengths
  {%
    \begingroup
      \xister@font
      \edef\xister@tmp
        {%
          \endgroup
          \xister@X=\the\dimexpr.5\dimexpr\xister@x\relax\relax
          \xister@Y=\the\dimexpr.5\dimexpr\xister@y\relax\relax
          \xister@Th=\the\dimexpr.5\dimexpr\xister@th\relax\relax
          \xister@Tv=\the\dimexpr.5\dimexpr\xister@tv\relax\relax
          \xister@Tdu=\the\dimexpr.5\dimexpr\xister@tdu\relax\relax
          \xister@Tdd=\the\dimexpr.5\dimexpr\xister@tdd\relax\relax
          \xister@Five@dot=\the\dimexpr\xister@five@dot\relax
          \xister@Six@dot=\the\dimexpr\xister@six@dot\relax
          \ifx\xister@ts\@empty
            \xister@Ts=%
              \ifx\xister@build\xister@build@h\xister@Th\else\xister@Tv\fi
          \else
            \xister@Ts=\the\dimexpr.5\dimexpr\xister@ts\relax\relax
          \fi
        }%
    \xister@tmp
  }
%    \end{macrocode}
% \end{macro}
%
% \begin{macro}[internal]{\xister@ensure@current}
%   The glyphs have to be rebuilt if the font size or colour changed. The latter
%   is an annoyance because we can build boxes in a way that they change colour
%   to the surrounding colour, but \pkg{PGF} is too intelligent here and does
%   issue the corresponding |\special|s to save the current colour for the box.
%   Because of that we test which |\f@size| and colour (using |\\color@.|) was
%   active when we last built the glyphs. If this matches we do nothing, else we
%   rebuild the glyphs.
%
%   The |\edef| is just used to remove the |debug| code if it isn't necessary.
%    \begin{macrocode}
\protected\edef\xister@ensure@current
  {%
    \unexpanded{\edef\xister@tmp{\xister@fontfeatures}}%
    \unexpanded{\unless\ifx\xister@tmp\xister@last@font}%
      \ifxister@dbg@
        \unexpanded
          {%
            \typeout
              {%
                Package xistercian Debug: Font changed. Building glyphs for
                \xister@tmp.
              }%
          }%
      \fi
    \unexpanded
      {%
          \let\xister@last@font\xister@tmp
          \xister@build
      }%
      \ifxister@dbg@
        \unexpanded{\typeout{Package xistercian Debug: Done.}}%
      \fi
    \unexpanded{\fi}%
  }
\AtBeginDocument{\xister@ensure@current}
%    \end{macrocode}
% \end{macro}
%
% \clearpage
% \PrintIndex
